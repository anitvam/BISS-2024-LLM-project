% This must be in the first 5 lines to tell arXiv to use pdfLaTeX, which is strongly recommended.
\pdfoutput=1
% In particular, the hyperref package requires pdfLaTeX in order to break URLs across lines.

\documentclass[11pt]{article}

% Change "review" to "final" to generate the final (sometimes called camera-ready) version.
% Change to "preprint" to generate a non-anonymous version with page numbers.
\usepackage[final]{acl}

% Standard package includes
\usepackage{times}
\usepackage{latexsym}

% For proper rendering and hyphenation of words containing Latin characters (including in bib files)
\usepackage[T1]{fontenc}
% For Vietnamese characters
% \usepackage[T5]{fontenc}
% See https://www.latex-project.org/help/documentation/encguide.pdf for other character sets

% This assumes your files are encoded as UTF8
\usepackage[utf8]{inputenc}

% This is not strictly necessary, and may be commented out,
% but it will improve the layout of the manuscript,
% and will typically save some space.
\usepackage{microtype}

% This is also not strictly necessary, and may be commented out.
% However, it will improve the aesthetics of text in
% the typewriter font.
\usepackage{inconsolata}

%Including images in your LaTeX document requires adding
%additional package(s)
\usepackage{graphicx}

\usepackage{acronym}
\usepackage[inline]{enumitem}
% MUST BE THE LAST IMPORT
\usepackage{cleveref}

\acrodef{acti}[ACTI]{Automatic Conspiracy Theory Identification}
\acrodef{llm}[LLM]{Large Language Model}

\newcommand{\meta}[1]{{\color{blue}#1}}

% If the title and author information does not fit in the area allocated, uncomment the following
%
%\setlength\titlebox{<dim>}
%
% and set <dim> to something 5cm or larger.

\title{Large Language Models Project \\
Bertinoro International Spring School (BISS) 2024}

% Author information can be set in various styles:
% For several authors from the same institution:
% \author{Martina Baiardi \and Davide Domini \and Nicolas Farabegoli \and Alessandro Petrella \and Gianni Tumedei }
%         Address line \\ ... \\ Address line}
% if the names do not fit well on one line use
%         Author 1 \\ {\bf Author 2} \\ ... \\ {\bf Author n} \\
% For authors from different institutions:
% \author{Author 1 \\ Address line \\  ... \\ Address line
%         \And  ... \And
%         Author n \\ Address line \\ ... \\ Address line}
% To start a separate ``row'' of authors use \AND, as in
% \author{Author 1 \\ Address line \\  ... \\ Address line
%         \AND
%         Author 2 \\ Address line \\ ... \\ Address line \And
%         Author 3 \\ Address line \\ ... \\ Address line}


\author{
  Martina Baiardi \\
  University of Bologna \\
  {\bf m.baiardi@unibo.it} \\ \And
  Davide Domini \\
  University of Bologna \\
  {\bf davide.domini@unibo.it} \\  \And
  Nicolas Farabegoli \\
  University of Bologna \\
  {\bf nicolas.farabegoli@unibo.it} \\  \AND 
  Alessandro Petrella\\
  University of Bologna \\ 
  {\bf alessandro.petrella@unibo.it} \\ \And 
  Gianni Tumedei \\
  University of Bologna \\
  {\bf gianni.tumedei2@unibo.it}
}

\begin{document}

\maketitle

\begin{abstract}
Recently,
social network platform such as Telegram, 4chain, and Parler do not provide strong moderation policies,
determining the proliferation of conspiracies theories in many contexts like COVID and war in Ukraine,
diffusing dangerous ideas and generating social harm.
%
In this paper we present the process for fine-tune two \acp{llm} architectures
on an Italian dataset coming from a EVALITA challenge
for identifying conspiracy theories in posts coming from Telegram.
%
We provide details about the two adopted architectures,
the model configuration and the training process adopted in our experiments.
%
\meta{Briefly summarise the finding of the experiments}
\end{abstract}

\section{Task description}\label{sec:task-description}
We opt-in for a comparative study between \emph{Encoder-only} and \emph{Decoder-only} architectures on the same task.
%
Specifically we identified the \emph{\ac{acti}}~\footnote{\url{https://russogiuseppe.github.io/ACTI/}} task,
that aims to individuate conspiracy theories coming from lax moderated policies platform like Telegram, 4chan, and Parler.
%
We focused on the \emph{Subtask A} where a system must recognise if a Telegram post is conspiratorial or not.
%
A post is defined conspiratorial if:
\begin{enumerate*}[label=(\roman{*})]
  \item express the belief that major events are controlled by and/or manipulated by powerful people protecting their interests; or
  \item interpretation of events meant to contribute to support conspiracy theories.
\end{enumerate*}
A sentence is considered conspiratorial even if it shares some claims intended to undermine commonly accepted views on societal issues.

\section{Dataset description}\label{sec:dataset-description}
The dataset for the \ac{acti}-A training is a CSV file containing three columns:
\begin{itemize}
  \item \textbf{id}: represents a unique identifier of the post;
  \item \textbf{comment\_text}: contains the raw text written in the post; and
  \item \textbf{conspiratorial}: represents a binary label where 0 indicates the post is not conspiratorial,
    while 1 indicates a conspiratorial post.
\end{itemize}
The whole dataset is split into two separate CSV files: one meant for training process,
and the second meant for testing purposes.
%
Notably,
the second dataset do not contain the \textbf{conspiratorial} column.

\section{Architecture overview}\label{sec:architecture-overview}

\section{Experimental setup}\label{sec:experimental-setup}

\subsection{Preprocessing}\label{sec:preprocessing}
\begin{figure*}
  \centering
  \includegraphics[width=\textwidth]{figures/class_distribution.pdf}
  \caption{TBD}
  \label{fig:class-frequency}
\end{figure*}

\begin{figure*}
  \centering
  \includegraphics[width=\textwidth]{figures/comment_length_distribution.pdf}
  \caption{
    Comment length distribution for both the training and test dataset.
    %
    On the upper charts the considered post are the raw ones coming from the original dataset;
    the lower charts a cleanup preprocessing is applied to the posts.
  }
  \label{fig:words-distribution}
\end{figure*}

\subsection{Model configuration}\label{sec:model-config}

\subsection{Training process}\label{sec:training-process}

\section{Result and analysis}\label{sec:results-analysis}

\nocite{*}

\bibliography{bibliography}

% \appendix

% \section{Example Appendix}
% \label{sec:appendix}

% This is an appendix.

\end{document}
